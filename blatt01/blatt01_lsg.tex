\documentclass[a4paper, 11pt]{article}
\usepackage{geometry}
\usepackage{graphicx}
\usepackage{a4wide}
\usepackage{ulem}
\usepackage{amsthm}
\usepackage{amsmath}
\usepackage{amsfonts}
\usepackage{amssymb}
\usepackage[T1]{fontenc}
\usepackage{german}
\usepackage{graphicx}
\usepackage{epic}
\usepackage{enumerate}
\usepackage [latin1]{inputenc}
\geometry{a4paper,left=25mm,right=25mm,top=10mm,bottom=15mm}
\renewcommand{\baselinestretch}{1.5}
\newcommand{\ol}{\overline}
\newcommand{\makeline}{\hrule\vspace{5pt}}
\newcommand{\ip}[2]{\left< #1, #2 \right>}

\title{1. �bungsblatt zu Software Qualit"at}
\author{Michel Meyer, Manuel Schwarz}

\begin{document}
  \maketitle

  \section*{Aufgabe 1.1}
	\subsection*{(a)}
	\begin{itemize}
		\item \textbf{Portabilit�t}: Plattformunabh�ngigkeit durch JAVA gegeben.
		\item \textbf{Vollst�ndigkeit}: W�hrend der Programmierung wurde sichergestellt, dass alle in der Aufgabe genannten Punkte implementiert wurden.
		\item \textbf{Benutzbarkeit}: Intuitive, traditionelle und schlichte GUI, die die Benutzung schnell und einfach h�lt.
	\end{itemize}
	\subsection*{(b)}
	\begin{itemize}
		\item 
	\end{itemize}
  \section*{Aufgabe 1.2}
  \begin{enumerate}[(a)]
    \item \begin{enumerate}[1.]
            \item Qualit�tsma� : MTFB
            \item Qualit�tsmerkmal: Zuverl�ssigkeit
            \item Auspr�gung: $>$ 500s
          \end{enumerate}
  \end{enumerate}
\end{document}




